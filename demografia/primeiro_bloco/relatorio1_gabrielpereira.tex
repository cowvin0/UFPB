% Options for packages loaded elsewhere
\PassOptionsToPackage{unicode}{hyperref}
\PassOptionsToPackage{hyphens}{url}
\PassOptionsToPackage{dvipsnames,svgnames,x11names}{xcolor}
%
\documentclass[
  12pt,
  letterpaper,
  DIV=11,
  numbers=noendperiod]{scrreprt}

\usepackage{amsmath,amssymb}
\usepackage{iftex}
\ifPDFTeX
  \usepackage[T1]{fontenc}
  \usepackage[utf8]{inputenc}
  \usepackage{textcomp} % provide euro and other symbols
\else % if luatex or xetex
  \usepackage{unicode-math}
  \defaultfontfeatures{Scale=MatchLowercase}
  \defaultfontfeatures[\rmfamily]{Ligatures=TeX,Scale=1}
\fi
\usepackage{lmodern}
\ifPDFTeX\else  
    % xetex/luatex font selection
\fi
% Use upquote if available, for straight quotes in verbatim environments
\IfFileExists{upquote.sty}{\usepackage{upquote}}{}
\IfFileExists{microtype.sty}{% use microtype if available
  \usepackage[]{microtype}
  \UseMicrotypeSet[protrusion]{basicmath} % disable protrusion for tt fonts
}{}
\usepackage{xcolor}
\setlength{\emergencystretch}{3em} % prevent overfull lines
\setcounter{secnumdepth}{-\maxdimen} % remove section numbering
% Make \paragraph and \subparagraph free-standing
\ifx\paragraph\undefined\else
  \let\oldparagraph\paragraph
  \renewcommand{\paragraph}[1]{\oldparagraph{#1}\mbox{}}
\fi
\ifx\subparagraph\undefined\else
  \let\oldsubparagraph\subparagraph
  \renewcommand{\subparagraph}[1]{\oldsubparagraph{#1}\mbox{}}
\fi


\providecommand{\tightlist}{%
  \setlength{\itemsep}{0pt}\setlength{\parskip}{0pt}}\usepackage{longtable,booktabs,array}
\usepackage{calc} % for calculating minipage widths
% Correct order of tables after \paragraph or \subparagraph
\usepackage{etoolbox}
\makeatletter
\patchcmd\longtable{\par}{\if@noskipsec\mbox{}\fi\par}{}{}
\makeatother
% Allow footnotes in longtable head/foot
\IfFileExists{footnotehyper.sty}{\usepackage{footnotehyper}}{\usepackage{footnote}}
\makesavenoteenv{longtable}
\usepackage{graphicx}
\makeatletter
\def\maxwidth{\ifdim\Gin@nat@width>\linewidth\linewidth\else\Gin@nat@width\fi}
\def\maxheight{\ifdim\Gin@nat@height>\textheight\textheight\else\Gin@nat@height\fi}
\makeatother
% Scale images if necessary, so that they will not overflow the page
% margins by default, and it is still possible to overwrite the defaults
% using explicit options in \includegraphics[width, height, ...]{}
\setkeys{Gin}{width=\maxwidth,height=\maxheight,keepaspectratio}
% Set default figure placement to htbp
\makeatletter
\def\fps@figure{htbp}
\makeatother
% definitions for citeproc citations
\NewDocumentCommand\citeproctext{}{}
\NewDocumentCommand\citeproc{mm}{%
  \begingroup\def\citeproctext{#2}\cite{#1}\endgroup}
\makeatletter
 % allow citations to break across lines
 \let\@cite@ofmt\@firstofone
 % avoid brackets around text for \cite:
 \def\@biblabel#1{}
 \def\@cite#1#2{{#1\if@tempswa , #2\fi}}
\makeatother
\newlength{\cslhangindent}
\setlength{\cslhangindent}{1.5em}
\newlength{\csllabelwidth}
\setlength{\csllabelwidth}{3em}
\newenvironment{CSLReferences}[2] % #1 hanging-indent, #2 entry-spacing
 {\begin{list}{}{%
  \setlength{\itemindent}{0pt}
  \setlength{\leftmargin}{0pt}
  \setlength{\parsep}{0pt}
  % turn on hanging indent if param 1 is 1
  \ifodd #1
   \setlength{\leftmargin}{\cslhangindent}
   \setlength{\itemindent}{-1\cslhangindent}
  \fi
  % set entry spacing
  \setlength{\itemsep}{#2\baselineskip}}}
 {\end{list}}
\usepackage{calc}
\newcommand{\CSLBlock}[1]{\hfill\break\parbox[t]{\linewidth}{\strut\ignorespaces#1\strut}}
\newcommand{\CSLLeftMargin}[1]{\parbox[t]{\csllabelwidth}{\strut#1\strut}}
\newcommand{\CSLRightInline}[1]{\parbox[t]{\linewidth - \csllabelwidth}{\strut#1\strut}}
\newcommand{\CSLIndent}[1]{\hspace{\cslhangindent}#1}

\usepackage{pdflscape}
\newcommand{\blandscape}{\begin{landscape}}
\newcommand{\elandscape}{\end{landscape}}
\KOMAoption{captions}{tableheading}
\makeatletter
\@ifpackageloaded{caption}{}{\usepackage{caption}}
\AtBeginDocument{%
\ifdefined\contentsname
  \renewcommand*\contentsname{Índice}
\else
  \newcommand\contentsname{Índice}
\fi
\ifdefined\listfigurename
  \renewcommand*\listfigurename{Lista de Figuras}
\else
  \newcommand\listfigurename{Lista de Figuras}
\fi
\ifdefined\listtablename
  \renewcommand*\listtablename{Lista de Tabelas}
\else
  \newcommand\listtablename{Lista de Tabelas}
\fi
\ifdefined\figurename
  \renewcommand*\figurename{Figura}
\else
  \newcommand\figurename{Figura}
\fi
\ifdefined\tablename
  \renewcommand*\tablename{Tabela}
\else
  \newcommand\tablename{Tabela}
\fi
}
\@ifpackageloaded{float}{}{\usepackage{float}}
\floatstyle{ruled}
\@ifundefined{c@chapter}{\newfloat{codelisting}{h}{lop}}{\newfloat{codelisting}{h}{lop}[chapter]}
\floatname{codelisting}{Listagem}
\newcommand*\listoflistings{\listof{codelisting}{Lista de Listagens}}
\makeatother
\makeatletter
\makeatother
\makeatletter
\@ifpackageloaded{caption}{}{\usepackage{caption}}
\@ifpackageloaded{subcaption}{}{\usepackage{subcaption}}
\makeatother
\ifLuaTeX
\usepackage[bidi=basic]{babel}
\else
\usepackage[bidi=default]{babel}
\fi
\babelprovide[main,import]{brazilian}
% get rid of language-specific shorthands (see #6817):
\let\LanguageShortHands\languageshorthands
\def\languageshorthands#1{}
\ifLuaTeX
  \usepackage{selnolig}  % disable illegal ligatures
\fi
\usepackage{bookmark}

\IfFileExists{xurl.sty}{\usepackage{xurl}}{} % add URL line breaks if available
\urlstyle{same} % disable monospaced font for URLs
\hypersetup{
  pdftitle={Relatório (escolher o nome depois)},
  pdfauthor={Gabriel de Jesus Pereira},
  pdflang={pt-br},
  colorlinks=true,
  linkcolor={blue},
  filecolor={Maroon},
  citecolor={Blue},
  urlcolor={Blue},
  pdfcreator={LaTeX via pandoc}}

\title{Relatório (escolher o nome depois)}
\usepackage{etoolbox}
\makeatletter
\providecommand{\subtitle}[1]{% add subtitle to \maketitle
  \apptocmd{\@title}{\par {\large #1 \par}}{}{}
}
\makeatother
\subtitle{Universidade Federal da Paraíba - CCEN}
\author{Gabriel de Jesus Pereira}
\date{29 de julho de 2024}

\begin{document}
\maketitle

\chapter{Introdução}\label{introduuxe7uxe3o}

O estado do Rio de Janeiro, localizado na região Sudeste do Brasil, é um
dos estados mais importantes e influentes do país. De acordo com o
último censo realizado, o de 2022, o estado do Rio de Janeiro como um
todo

\chapter{Metodologia}\label{metodologia}

\section{Recursos computacionais}\label{recursos-computacionais}

~~~As análises a seguir foram realizadas utilizando a linguagem de
programação R (R CORE TEAM, 2024) com o conjunto de pacotes tidyverse
(WICKHAM \emph{et al.}, 2019) para ciência de dados, utilizando
principalmente o pacote ggplot2 (WICKHAM, 2016) para visualização de
dados. Além disso, os documentos do relatório foram feitos com o Quarto
(ALLAIRE \emph{et al.}, 2022), um sistema de escrita e publicação
científica, e os códigos utilizados estão disponíveis no GitHub (J.
PEREIRA, 2024).

\section{Obtenção dos dados}\label{obtenuxe7uxe3o-dos-dados}

~~~A obtenção dos dados demográficos foi realizada através de duas
fontes. Os dados de mortalidade e natalidade foram obtidos através do
TABNET, desenvolvido pelo DATASUS. O TABNET é um tabulador genérico de
domínio público que permite organizar dados de forma rápida, conforme a
consulta que se deseja tabular, e o DATASUS providencia informações que
podem servir para analisar saúde pública e variáveis demográficas que
ajudam na elaboração de programas de ações de saúde. Além disso, os
dados referentes à quantidade da população do estado do Rio de Janeiro
foram obtidos atráveis de estimativas do Instituto Brasileiro de
Geografia e Estatística - IBGE.

\vspace{12pt}

~~~A análise considerada abrange o período de 2010 a 2020, portanto, os
dados de população, mortalidade e natalidade estão delimitados nesse
intervalo. Os dados de mortalidade estão detalhados por município dentro
do estado do Rio de Janeiro. As projeções populacionais do IBGE incluem
faixas etárias e são divididas por sexo, o que permite a análise da
pirâmide etária do estado do Rio de Janeiro. Além disso, os dados de
natalidade estão categorizados por diferentes intervalos de nascidos
vivos, proporcionando uma visão detalhada dos padrões de natalidade na
região.

\section{Análise Exploratória de
Dados}\label{anuxe1lise-exploratuxf3ria-de-dados}

~~~A análise exploratória de dados é uma etapa fundamental em qualquer
estudo que utilize a estatística como principal ferramenta de análise.
Ela permite identificar padrões de comportamento nos dados e descobrir
relações entre as variáveis estudadas. Assim, após a coleta e
organização dos dados, a primeira etapa deste estudo foi a análise
exploratória de dados. Essa etapa possibilitou a análise dos
comportamentos de natalidade, mortalidade e crescimento populacional.
Para identificar esses diferentes comportamentos, foram elaborados
gráficos e tabelas.

\section{População}\label{populauxe7uxe3o}

\section{Natalidade}\label{natalidade}

\section{Mortalidade}\label{mortalidade}

\chapter{Resultados}\label{resultados}

\section{Descritiva dos dados}\label{descritiva-dos-dados}

\chapter*{Conclusão}\label{conclusuxe3o}
\addcontentsline{toc}{chapter}{Conclusão}

\phantomsection\label{refs}
\begin{CSLReferences}{0}{1}
\bibitem[\citeproctext]{ref-quarto}
ALLAIRE, J. J. \emph{et al.} Quarto. 2022. Disponível em:
\textless{}\url{https://quarto.org}\textgreater.

\bibitem[\citeproctext]{ref-github}
J. PEREIRA, G. De. Códigos da análise demográfica para o primeiro
relatório da disciplina de demografia. 2024. Disponível em:
\textless{}\url{https://github.com/cowvin0/UFPB/tree/main/demografia/primeiro_bloco}\textgreater.

\bibitem[\citeproctext]{ref-rlang}
R CORE TEAM. \textbf{\href{https://www.R-project.org/}{R: A Language and
Environment for Statistical Computing}}. Vienna, Austria: R Foundation
for Statistical Computing, 2024.

\bibitem[\citeproctext]{ref-ggplot}
WICKHAM, H. \textbf{\href{https://ggplot2.tidyverse.org}{ggplot2:
Elegant Graphics for Data Analysis}}. {[}S.l.{]}: Springer-Verlag New
York, 2016.

\bibitem[\citeproctext]{ref-tidy}
\_\_\_\_\_\_ \emph{et al.}
\href{https://doi.org/10.21105/joss.01686}{Welcome to the tidyverse}.
\textbf{Journal of Open Source Software}, 2019. v. 4, n. 43, p. 1686.

\end{CSLReferences}



\end{document}
