% Options for packages loaded elsewhere
\PassOptionsToPackage{unicode}{hyperref}
\PassOptionsToPackage{hyphens}{url}
\PassOptionsToPackage{dvipsnames,svgnames,x11names}{xcolor}
%
\documentclass[
  12pt,
  a4paper,
]{scrreprt}

\usepackage{amsmath,amssymb}
\usepackage{iftex}
\ifPDFTeX
  \usepackage[T1]{fontenc}
  \usepackage[utf8]{inputenc}
  \usepackage{textcomp} % provide euro and other symbols
\else % if luatex or xetex
  \usepackage{unicode-math}
  \defaultfontfeatures{Scale=MatchLowercase}
  \defaultfontfeatures[\rmfamily]{Ligatures=TeX,Scale=1}
\fi
\usepackage{lmodern}
\ifPDFTeX\else  
    % xetex/luatex font selection
\fi
% Use upquote if available, for straight quotes in verbatim environments
\IfFileExists{upquote.sty}{\usepackage{upquote}}{}
\IfFileExists{microtype.sty}{% use microtype if available
  \usepackage[]{microtype}
  \UseMicrotypeSet[protrusion]{basicmath} % disable protrusion for tt fonts
}{}
\usepackage{xcolor}
\usepackage[left=3cm,,right=2cm,,top=3cm,,bottom=2cm]{geometry}
\setlength{\emergencystretch}{3em} % prevent overfull lines
\setcounter{secnumdepth}{5}


\providecommand{\tightlist}{%
  \setlength{\itemsep}{0pt}\setlength{\parskip}{0pt}}\usepackage{longtable,booktabs,array}
\usepackage{calc} % for calculating minipage widths
% Correct order of tables after \paragraph or \subparagraph
\usepackage{etoolbox}
\makeatletter
\patchcmd\longtable{\par}{\if@noskipsec\mbox{}\fi\par}{}{}
\makeatother
% Allow footnotes in longtable head/foot
\IfFileExists{footnotehyper.sty}{\usepackage{footnotehyper}}{\usepackage{footnote}}
\makesavenoteenv{longtable}
\usepackage{graphicx}
\makeatletter
\newsavebox\pandoc@box
\newcommand*\pandocbounded[1]{% scales image to fit in text height/width
  \sbox\pandoc@box{#1}%
  \Gscale@div\@tempa{\textheight}{\dimexpr\ht\pandoc@box+\dp\pandoc@box\relax}%
  \Gscale@div\@tempb{\linewidth}{\wd\pandoc@box}%
  \ifdim\@tempb\p@<\@tempa\p@\let\@tempa\@tempb\fi% select the smaller of both
  \ifdim\@tempa\p@<\p@\scalebox{\@tempa}{\usebox\pandoc@box}%
  \else\usebox{\pandoc@box}%
  \fi%
}
% Set default figure placement to htbp
\def\fps@figure{htbp}
\makeatother

\usepackage{dirtree}
\usepackage{amsmath}
\usepackage[explicit]{titlesec}
\usepackage{setspace}
\usepackage{caption}
\usepackage[hang,flushmargin]{footmisc}
\usepackage{etoolbox}
\usepackage[shortlabels]{enumitem}
\usepackage{booktabs}
\usepackage{ragged2e}
\usepackage{pdflscape}
\usepackage{fancyhdr}

\newcommand{\blandscape}{\begin{landscape}}
\newcommand{\elandscape}{\end{landscape}}
\renewcommand{\footnotesize}{\scriptsize}
\renewcommand{\footnoterule}{\noindent\rule{5cm}{0.4pt}\vspace{0.2cm}}
\setlength{\footnotesep}{0.5em}

\setlength{\parskip}{0.0em}
\AtBeginEnvironment{quote}{\setstretch{1.0}}

\titleformat{\section}{\normalfont\large\bfseries}{}{0pt}{\thesection\quad#1}[]
\titleformat{\subsection}{\normalfont\normalsize\bfseries}{}{0pt}{\thesubsection\quad#1}[]
\titleformat{\subsubsection}{\normalfont\normalsize\itshape}{}{0pt}{#1}[]
\titleformat{\paragraph}[runin]{\normalfont\normalsize\bfseries}{}{0pt}{#1}[]
\titleformat{\subparagraph}[runin]{\normalfont\normalsize\itshape}{}{0pt}{#1}[]

\titlespacing*{\section}{0pt}{20pt}{10pt}
\titlespacing*{\subsection}{0pt}{15pt}{10pt}
\titlespacing*{\subsubsection}{0pt}{10pt}{10pt}
\titlespacing*{\paragraph}{0pt}{10pt}{10pt}
\titlespacing*{\subparagraph}{0pt}{10pt}{10pt}
\makeatletter
\@ifpackageloaded{caption}{}{\usepackage{caption}}
\AtBeginDocument{%
\ifdefined\contentsname
  \renewcommand*\contentsname{Índice}
\else
  \newcommand\contentsname{Índice}
\fi
\ifdefined\listfigurename
  \renewcommand*\listfigurename{Lista de Figuras}
\else
  \newcommand\listfigurename{Lista de Figuras}
\fi
\ifdefined\listtablename
  \renewcommand*\listtablename{Lista de Tabelas}
\else
  \newcommand\listtablename{Lista de Tabelas}
\fi
\ifdefined\figurename
  \renewcommand*\figurename{Figura}
\else
  \newcommand\figurename{Figura}
\fi
\ifdefined\tablename
  \renewcommand*\tablename{Tabela}
\else
  \newcommand\tablename{Tabela}
\fi
}
\@ifpackageloaded{float}{}{\usepackage{float}}
\floatstyle{ruled}
\@ifundefined{c@chapter}{\newfloat{codelisting}{h}{lop}}{\newfloat{codelisting}{h}{lop}[chapter]}
\floatname{codelisting}{Listagem}
\newcommand*\listoflistings{\listof{codelisting}{Lista de Listagens}}
\makeatother
\makeatletter
\makeatother
\makeatletter
\@ifpackageloaded{caption}{}{\usepackage{caption}}
\@ifpackageloaded{subcaption}{}{\usepackage{subcaption}}
\makeatother
\makeatletter
\@ifpackageloaded{tcolorbox}{}{\usepackage[skins,breakable]{tcolorbox}}
\makeatother
\makeatletter
\@ifundefined{shadecolor}{\definecolor{shadecolor}{rgb}{.97, .97, .97}}{}
\makeatother
\makeatletter
\@ifundefined{codebgcolor}{\definecolor{codebgcolor}{HTML}{F0F2F4}}{}
\makeatother
\makeatletter
\ifdefined\Shaded\renewenvironment{Shaded}{\begin{tcolorbox}[boxrule=0pt, breakable, frame hidden, colback={codebgcolor}, sharp corners, enhanced]}{\end{tcolorbox}}\fi
\makeatother

\ifLuaTeX
\usepackage[bidi=basic]{babel}
\else
\usepackage[bidi=default]{babel}
\fi
\babelprovide[main,import]{brazilian}
% get rid of language-specific shorthands (see #6817):
\let\LanguageShortHands\languageshorthands
\def\languageshorthands#1{}
\usepackage{bookmark}

\IfFileExists{xurl.sty}{\usepackage{xurl}}{} % add URL line breaks if available
\urlstyle{same} % disable monospaced font for URLs
\hypersetup{
  pdftitle={Primeiro relatório da disciplina de demografia II - Roraima},
  pdfauthor={Gabriel de Jesus Pereira},
  pdflang={pt-br},
  colorlinks=true,
  linkcolor={blue},
  filecolor={Maroon},
  citecolor={Blue},
  urlcolor={Blue},
  pdfcreator={LaTeX via pandoc}}


\title{Primeiro relatório da disciplina de demografia II - Roraima}
\author{Gabriel de Jesus Pereira}
\date{março, 2025}

\begin{document}
\cleardoublepage
\thispagestyle{empty}
{\centering
\noindent\rule{\textwidth}{0.5pt}

\vspace{2ex}

{\Large\bfseries Universidade Federal da Paraíba \par}
\vspace{1ex}
{\Large\bfseries Centro de Ciências Exatas e da Natureza \par}
\vspace{1ex}
{\Large\bfseries Departamento de Estatística \par}

\vfill

{\large\bfseries Primeiro relatório da disciplina de demografia II -
Roraima \par}

\vfill

{\large Gabriel de Jesus Pereira \par}
\vfill
{\normalsize março, 2025 \par}


\noindent\rule{\textwidth}{0.5pt}

}
\renewcommand*\contentsname{\centering Sumário \thispagestyle{empty}}
{
\hypersetup{linkcolor=}
\setcounter{tocdepth}{2}
\tableofcontents
}

\pagenumbering{arabic}
\pagestyle{fancy}

\fancyhf{}
\fancyhead[RO, LE]{\thepage}
\fancyhead[LO]{\leftmark}
\fancyhead[RE]{\thepage}

\fancypagestyle{plain}{
  \pagestyle{fancy}
  \fancyhf{}
  \fancyhead[RO, LE]{\thepage}
  \fancyhead[RE]{\thepage}
  \renewcommand{\headrulewidth}{0pt}
}

\chapter{Introdução}\label{introduuxe7uxe3o}

\chapter{Metodologia}\label{metodologia}

\section{Métodos para estimação da cobertura de nascidos
vivos}\label{muxe9todos-para-estimauxe7uxe3o-da-cobertura-de-nascidos-vivos}

~~~A estimação do número de nascidos vivos é essencial para a análise
demográfica e epidemiológica, especialmente em contextos onde há
subnotificação ou inconsistências nos registros civis. Diversos métodos
podem ser utilizados para avaliar a cobertura dos nascimentos, como
comparações entre fontes de dados, modelagem estatística e ajustes
baseados em fatores demográficos. Nesta seção, serão apresentadas as
principais técnicas utilizadas para essa estimação.

\subsection{Razão de sexo dos nascimentos
(RSN)}\label{razuxe3o-de-sexo-dos-nascimentos-rsn}

~~~A Razão de Sexo dos Nascimentos (RSN) é um indicador que expressa a
relação entre o número de nascidos vivos do sexo masculino e feminino em
uma população. Geralmente, espera-se que essa razão esteja em torno de
105, indicando um leve predomínio de nascimentos masculinos sobre os
femininos.

\vspace{12pt}

O cálculo da RSN é feito pela seguinte fórmula:

\[
RSN = \frac{N_M}{N_F} \times 100 \text{,}
\] em que \(N_M\) representa o número de nascidos vivos do sexo
masculino e \(N_F\) do sexo feminino.

\vspace{12pt}

Valores significativamente diferentes do esperado podem indicar
problemas na qualidade dos dados, como erros de registro ou
subnotificação diferenciada por sexo.

\vspace{12pt}

Os limites do intervalo de confiança a \(95\%\) podem ser calculados a
partir da seguinte expressão:

\[
\left[x, y\right] = p_{M} \pm 1,96 \sqrt{\frac{p_{M}p_{F}}{n}} \text{,}
\] em que \(n\) é o número total de nascimentos, \(p_{M}\) é a proporção
de nascidos vivos do sexo masculino e \(p_{F}\) do sexo feminino.

\vspace{12pt}

Por fim, para verificar se a qualidade de registro de nascimentos é boa,
basta verificar se o resultado das relações de sexo está incluso no
intervalo \(\left[a, b\right]\):

\[
a = \frac{x}{1 - x} \times 100 \text{ e } b = \frac{y}{1-y}\times 100\text{.}
\]

\subsection{Método que utiliza a equação básica do crescimento
populacional}\label{muxe9todo-que-utiliza-a-equauxe7uxe3o-buxe1sica-do-crescimento-populacional}

O método que utiliza a equação básica do crescimento populacional é
bastante simples, principalmente por assumir que a popuação é fechada.
Dessa forma, a estimativa de nascidos vivos será expresso pela equação:

\[
N_t = P_n - P_{0} + O_t
\] em que \(P_n\) são os nascidos vivos no instante \(n\), P\_\{0\} no
instante inicial e \(O_t\) os óbitos no período de estudo.

\vspace{12pt}

A partir dessa expressão, estima-se a cobertura dos nascimentos da
seguinte forma:

\[
\text{Cobertura dos nascimentos} = \frac{\text{Nascimentos registrados}\left(t\right)}{\text{Nascimentos esperados}\left(t\right)} \times 100
\]

\subsection{Método que faz uso das taxas de
fecundidade}\label{muxe9todo-que-faz-uso-das-taxas-de-fecundidade}

Neste método serão utilizadas as taxas de fecundidade do estado de
Roraima. Para estimar a cobertura nesse método, será utilizada a
seguinte expressão:

\[
C_i = \frac{NV_{obs}\left(i\right)}{NV_{est}\left(i\right)} \text{,}
\] em que \(NV_{obs}\left(i\right)\) é o total de nascidos vivos
observados na região \(i\) e \(NV_{est}\left(i\right)\) é o total de
nascidos vivos estimados na região \(i\).

\vspace{12pt}

Para encontrar a estimativa dos nascidos vivos
\(NV_{est}\left(i\right)\), será utilizado a taxa de fecundidade do
estado de Roraima, que pode ser encontrado a partir da seguinte
expressão:

\[
NV_{est} = \sum^{49}_{j=15} TEF_{j}\left(i\right) \times TM_{j} \text{,}
\] em que \(TEF_{j}\left(i\right)\) é a taxa específica de fecundidade
na faixa etária quinquenal \(j\) da região \(i\) e \(TM_j\) é o total de
mulheres na faixa etária quinquenal \(j\) da região \(j\).

\subsection{Método que faz uso da informação do SINASC e do
IBGE}\label{muxe9todo-que-faz-uso-da-informauxe7uxe3o-do-sinasc-e-do-ibge}

Aqui são utilizadas as estimativas de nascidos vivos fornecidos pelo
IBGE e os nascidos vivos fornecidos pelo SINASC. Por fim, para estimar a
cobertura, basta calcular utilizar a seguinte expressão:

\[
C_i = \frac{NV_{obs}\left(i\right)}{NV_{est}\left(i\right)}
\]

\section{Métodos para estimação da cobertura de
óbito}\label{muxe9todos-para-estimauxe7uxe3o-da-cobertura-de-uxf3bito}

A cobertura dos óbitos refere-se à proporção de mortes registradas em
relação ao total de óbitos ocorridos em uma população. Em contextos onde
há subnotificação ou falhas nos sistemas de informação, diferentes
métodos são empregados para estimar a verdadeira magnitude da
mortalidade.

\vspace{12pt}

Entre as principais abordagens utilizadas, destacam-se os métodos
demográficos indiretos, como o método de Brass, que utiliza informações
da estrutura etária da população e da mortalidade infantil para estimar
a cobertura. Além disso, comparações entre diferentes bases de dados,
modelagem estatística e técnicas de reconciliação de fontes são
amplamente empregadas para corrigir deficiências nos registros.

\vspace{12pt}

Esses métodos são fundamentais para garantir a confiabilidade dos
indicadores de mortalidade e subsidiar políticas públicas voltadas à
saúde e ao planejamento populacional.

\subsection{Método que faz uso da equação básica do crescimento
populacional}\label{muxe9todo-que-faz-uso-da-equauxe7uxe3o-buxe1sica-do-crescimento-populacional}

De forma semelhante ao método de cobertura de nascidos vivos que faz uso
da equação básica do crescimento populacional, para estimar os óbitos,
basta isolar a sua componente:

\[
O_t = N_t + P_{0} - P_n
\]

Por fim, para estimar a cobertura de óbitos, utiliza-se a seguinte
expressão:

\[
\text{Cobertura dos óbitos} = \frac{\text{Óbitos registrados}\left(t\right)}{\text{Óbitos esperados}\left(t\right)} \times 100
\]

\subsection{Método da Equação do Balanço de Crescimento de
Brass}\label{muxe9todo-da-equauxe7uxe3o-do-balanuxe7o-de-crescimento-de-brass}

Esse método avalia a cobertura de óbitos da população a partir dos cinco
anos de idade. Para fazer sua aplicação, é necessário considerar a
população estável, a cobertura de óbitos é constante por idade a partir
dos 5 anos e as distribuições por idade da população não devem conter
erros de declaração.

\vspace{12pt}

Nesse método é utilizado uma regressão linear, a partir da qual será
estimada o fator de correção dos óbitos \(\left(k\right)\), a taxa de
crescimento da população estável \(\left(r\right)\). Dessa forma, essa
regressão linear terá a seguinte relação:

\[
\frac{N\left(a\right)}{N\left(a+\right)} = r + k\frac{D^{'}\left(a+\right)}{N\left(a+\right)}\text{,}
\] em que \(N\left(a\right)\) é a população exata na idade \(a\),
\(N\left(a+\right)\) o somatório de pessoas que estão na idade exata até
um limite de idades \(w\) qualquer e \(D^{'}\left(a+\right)\) são os
óbitos registrados e afetados por erros na idade \(a+\).

\vspace{12pt}

Após o ajuste da regressão linear, a cobertura dos óbitos será dada por:

\[
C = \frac{1}{k}\text{,}
\] em que se o fator de correção for \(k > 1\) implica em sub-registro e
\(k<1\) implica em sobre-registro.

\subsection{Método de Leadermann para
redistribuição}\label{muxe9todo-de-leadermann-para-redistribuiuxe7uxe3o}

O método de Leadermann é uma abordagem utilizada para redistribuir
óbitos classificados com causas mal definidas entre categorias
específicas de mortalidade. Essa técnica busca minimizar o impacto da
subnotificação e da imprecisão nos registros, permitindo uma estimativa
mais fiel da estrutura de mortalidade de uma população.

\vspace{12pt}

A redistribuição é feita com base na suposição de que a proporção de
óbitos por causas definidas segue um padrão semelhante ao dos óbitos mal
definidosa. Assim, os óbitos mal classificados são redistribuídos
proporcionalmente entre as categorias bem definidas, considerando a
estrutura observada nos registros mais completos. Isso é feito
utilizando-se uma regressão linear

\vspace{12pt}

A equação de redistribuição dos óbitos por causas mal definidas para uma
determinada área é dada por:

\[
O_j = Y_j - X \beta_{j}
\] em que \(O_j\) são os óbitos redistribuídos da causa \(j\), \(Y_j\)
os óbitos observados da causa \(j\), \(\beta_j\) é o fator de
redistribuição da causa \(j\) e \(X\) são os óbitos da causa mal
definida.

\chapter{Resultado}\label{resultado}

\section{Resultado da estimação de cobertura de
nascimentos}\label{resultado-da-estimauxe7uxe3o-de-cobertura-de-nascimentos}

Nesta seção serão apresentadas cada uma das técnicas utilizadas para
analisar e estimar a cobertura dos nascidos vivos. O primeiro deles será
o RSN, depois o método de estimação utilizando a equação do crescimento
populacional, o que o utiliza as taxas de fecundidade e por último o que
utiliza as estimativas do IBGE e os dados do SINASC.

\subsection{Razão de Sexo dos Nascimentos
(RSN)}\label{razuxe3o-de-sexo-dos-nascimentos-rsn-1}

A abaixo apresenta os valores da Razão de Sexo dos Nascimentos (RSN)
para os anos de 2010 e 2020. Esse indicador expressa a relação entre os
nascidos vivos do sexo masculino e feminino, sendo esperado um valor em
torno de 105 em condições normais.

\begin{longtable}[]{@{}lllllllll@{}}
\caption{Resultados da Razão de sexo dos nascimentos para o ano de 2010
e 2020}\label{T_7daf9}\tabularnewline
\toprule\noalign{}
Ano & Masculino & Feminino & RSN & total & x & y & a & b \\
\midrule\noalign{}
\endfirsthead
\toprule\noalign{}
Ano & Masculino & Feminino & RSN & total & x & y & a & b \\
\midrule\noalign{}
\endhead
\bottomrule\noalign{}
\endlastfoot
2010 & 4990 & 4748 & 105.097 & 9738 & 0.507 & 0.518 & 102.974 &
107.265 \\
2020 & 7075 & 6684 & 105.850 & 13759 & 0.510 & 0.519 & 104.044 &
107.688 \\
\end{longtable}

Em 2010, a RSN foi de 105,097, indicando que para cada 100 meninas
nasceram aproximadamente 105 meninos. Já em 2020, esse valor aumentou
para 105,850, sugerindo um leve crescimento na proporção de nascimentos
masculinos em relação aos femininos. Além disso, ao observar o intervalo
\(\left[a, b\right]\), tem-se que há uma boa qualidade no registro dos
dados, pois os valores estimados da \(RSN\) estão contidos dentro do
intervalo.

\subsection{Método que utiliza a equação básica do crescimento
populacional}\label{muxe9todo-que-utiliza-a-equauxe7uxe3o-buxe1sica-do-crescimento-populacional-1}

\begin{longtable}[]{@{}llllll@{}}
\caption{}\label{T_fb292}\tabularnewline
\toprule\noalign{}
Ano & Óbito & População & nasc\_total & nascimento\_esperado &
cobertura\_estimada \\
\midrule\noalign{}
\endfirsthead
\toprule\noalign{}
Ano & Óbito & População & nasc\_total & nascimento\_esperado &
cobertura\_estimada \\
\midrule\noalign{}
\endhead
\bottomrule\noalign{}
\endlastfoot
2020 & 3580 & 563000 & 13759 & 117741.000 & 19.957 \\
2010 & 1640 & 450479 & 9738 & 117741.000 & 19.957 \\
\end{longtable}

\blandscape

\subsection{Método que faz uso das taxas de
fecundidade}\label{muxe9todo-que-faz-uso-das-taxas-de-fecundidade-1}

\begin{longtable}[]{@{}lllllllll@{}}
\caption{}\label{T_5130a}\tabularnewline
\toprule\noalign{}
Idade & Mulheres\_2000 & Mulheres\_2010 & TFE\_2000 & TFE\_2010 &
NV\_2000 & NV\_2010 & C\_2000 & C\_2010 \\
\midrule\noalign{}
\endfirsthead
\toprule\noalign{}
Idade & Mulheres\_2000 & Mulheres\_2010 & TFE\_2000 & TFE\_2010 &
NV\_2000 & NV\_2010 & C\_2000 & C\_2010 \\
\midrule\noalign{}
\endhead
\bottomrule\noalign{}
\endlastfoot
15 a 19 anos & 20760 & 25260 & 0.159 & 0.111 & 10897.410 & 11088.013 &
0.894 & 0.878 \\
20 a 24 anos & 16294 & 21788 & 0.214 & 0.150 & 10897.410 & 11088.013 &
0.894 & 0.878 \\
25 a 29 anos & 13430 & 21792 & 0.167 & 0.121 & 10897.410 & 11088.013 &
0.894 & 0.878 \\
30 a 34 anos & 11606 & 18669 & 0.100 & 0.081 & 10897.410 & 11088.013 &
0.894 & 0.878 \\
35 a 39 anos & 10175 & 14839 & 0.055 & 0.045 & 10897.410 & 11088.013 &
0.894 & 0.878 \\
40 a 44 anos & 7926 & 12269 & 0.018 & 0.015 & 10897.410 & 11088.013 &
0.894 & 0.878 \\
45 a 49 anos & 5710 & 10379 & 0.002 & 0.002 & 10897.410 & 11088.013 &
0.894 & 0.878 \\
\end{longtable}

\elandscape

\subsection{Método que faz uso dos dados do IBGE e
SINASC}\label{muxe9todo-que-faz-uso-dos-dados-do-ibge-e-sinasc}

\begin{longtable}[]{@{}lllll@{}}
\toprule\noalign{}
& Ano & IBGE & SINASC & Cobertura \\
\midrule\noalign{}
\endhead
\bottomrule\noalign{}
\endlastfoot
0 & 2020 & 13991 & 13760 & 0.983489 \\
1 & 2010 & 10686 & 9738 & 0.911286 \\
\end{longtable}

\section{Resultados da estimação de cobertura de
óbitos}\label{resultados-da-estimauxe7uxe3o-de-cobertura-de-uxf3bitos}

\subsection{Método que utiliza a equação básica do cresimento
populacional}\label{muxe9todo-que-utiliza-a-equauxe7uxe3o-buxe1sica-do-cresimento-populacional}

\subsection{Método da equação do balanço de crescimento de
Brass}\label{muxe9todo-da-equauxe7uxe3o-do-balanuxe7o-de-crescimento-de-brass-1}

\begin{verbatim}
Year: 2010, Gender: homens, Slope: 2.1068, Intercept: 0.0229, R²: 0.9420
Year: 2010, Gender: mulheres, Slope: 2.4507, Intercept: 0.0278, R²: 0.9089
Year: 2022, Gender: homens, Slope: 1.7873, Intercept: 0.0157, R²: 0.9605
Year: 2022, Gender: mulheres, Slope: 2.0700, Intercept: 0.0214, R²: 0.8932
\end{verbatim}

\subsection{Método de Leadermann}\label{muxe9todo-de-leadermann}




\end{document}
